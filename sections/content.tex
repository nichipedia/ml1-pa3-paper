%hg \section{content}
\setlength{\parindent}{10ex}

\textbf{Nicholas Moran} |
\textbf{Machine Learning 1 Spring 19} |
\textbf{Programming Assignment 3} 

\section{Voice Features for detecting Parkinson's Disease}
    Detecting Parkinson's Disease can be difficult. 
    It takes time and many visits to correctly identify the early stages of motor function impairments.
    Detecting Parkinson's early can drastically help with its treatment.
    One of the earliest indicators of Parkinson's is vocal impairment.
    This means that detecting vocal impairments that are linked to Parkinson's could be a way to detect Parkinson's  earlier.
    People with Parkinson's typically display a variety of vocal symptoms that include issues with vowel sounds and articulating speech \cite{little}.
    Issues with producing vowels (\textit{dysphinia}) can include: volume control, breathiness, roughness, and exaggerated vocal tremor \cite{little}.
    These issues can be represented as features in a data-set, and then used to predict Parkinson's.


\section{Feature Extraction}
    The features were generated from human voice audio recordings \cite{little}.
    These recordings were created in a controlled setting with identical microphones across genders and age ranges.
    Along with varying progression stages of the disease. 
    Several algorithms for extracting features were applied to this data-set.
    The authors implemented a \textit{Fractal Scaling Analysis} model for detecting hoarseness with great effect \cite{McSharry}.
    As stated above, Dysphonia features are potential predictors. 
    The authors used \textit{Nonlinear Time Series Analysis} tools for extracting these features.
    Tools such as correlation dimensions, recurrence period density entropy, and detrended fluctuation analysis were used for this extraction.





\section{Model Performance}

\begin{center}
    \begin{tabular}{|c|c|c|c|}
        \hline
            \multicolumn{1}{|r|}{} & \multicolumn{3}{|c|}{Error} \\
        \hline
        \textbf{Cross Validation} & \textbf{LR} & \textbf{LDA} & \textbf{QDA} \\
        \hline
        CV1 & 0.15 & 0.15 & 0.25 \\
        \hline
        CV2 & 0.05 & 0.05 & 0.2 \\
        \hline
        CV3 & 0.15 & 0.1 & 0.7 \\
        \hline 
        CV4 & 0.25 & 0.3 & 0.55 \\
        \hline 
        CV5 & 0.1 & 0.1 & 0.15 \\
        \hline
        CV6 & 0.0 & 0.0 & 0.157 \\
        \hline 
        CV7 & 0.105 & 0.105 & 0.105 \\
        \hline 
        CV8 & 0.157 & 0.210 & 0.105 \\
        \hline
        CV9 & 0.105 & 0.105 & 0.263 \\
        \hline
        CV10 & 0.105 & 0.210 & 0.683 \\
        \hline
        \textbf{Average} & 0.117 & 0.133 & 0.369 \\
        \hline
    \end{tabular}
\end{center}


\section{Code}
I implemented the algorithms using Python with numpy.
The implementations can be found the Classification.py file. 
Each implementation extends sklearns BaseEstimator class. 
This is so my implementations follow the sklearn api.
This allows me to use the tools provided by the framework.
I produce my results in the parkinsons-models.py file.
For the 10 fold cross validation I use sklearns cross validation class.\\
\textbf{Python Dependencies}
\begin{itemize}
    \item numpy
    \item pandas
    \item sklearn
\end{itemize}
\textbf{Instructions for running code}
\begin{itemize}
    \item pip install python dependencies
    \item open a terminal and cd to the project directory
    \item run the parkinsons-models.py class
    \item the results will be printed to the console
\end{itemize}
